\documentclass[12pt,a4paper]{article}
\usepackage[utf8]{inputenc}
\usepackage[french]{babel}
\usepackage{geometry}
\geometry{margin=2.5cm}
\usepackage{setspace}
\usepackage{array}
\onehalfspacing

\begin{document}

\section{Analyse des Besoins et Specifications}

\subsection{Methodologie d'Analyse}

L'analyse des besoins a ete menee selon une approche systematique impliquant plusieurs parties prenantes : chercheurs en imagerie medicale, techniciens de laboratoire, et utilisateurs finaux. Cette analyse a combine entretiens individuels, observations sur site, et analyse des workflows existants.

\subsection{Parties Prenantes}

\subsubsection{Utilisateurs Primaires}
\begin{itemize}
\item \textbf{Chercheurs en histologie} : Besoin d'outils precis pour l'analyse scientifique
\item \textbf{Techniciens de laboratoire} : Recherche d'efficacite et de simplicite d'utilisation
\item \textbf{Pathologistes} : Exigences de precision diagnostique
\end{itemize}

\subsubsection{Utilisateurs Secondaires}
\begin{itemize}
\item \textbf{Administrateurs IT} : Contraintes de deploiement et maintenance
\item \textbf{Etudiants et stagiaires} : Besoin d'apprentissage rapide
\end{itemize}

\subsection{Besoins Fonctionnels Detailles}

\subsubsection{Gestion des Images}

\begin{table}[h]
\centering
\begin{tabular}{|p{4cm}|p{2cm}|p{6cm}|}
\hline
\textbf{Fonctionnalite} & \textbf{Priorite} & \textbf{Description Detaillee} \\
\hline
Chargement TIFF pyramidal & Critique & Support complet des fichiers TIFF multi-resolution avec preservation des metadonnees \\
\hline
Support format SVS & Haute & Compatibilite avec les fichiers Aperio ScanScope pour l'interoperabilite \\
\hline
Chargement par lots & Moyenne & Capacite de charger plusieurs fragments simultanement \\
\hline
Apercu rapide & Moyenne & Generation automatique de miniatures pour navigation \\
\hline
\end{tabular}
\caption{Besoins fonctionnels - Gestion des images}
\end{table}

\subsubsection{Manipulation des Fragments}

Les besoins de manipulation incluent :

\begin{itemize}
\item \textbf{Transformations geometriques} : Rotation libre (pas seulement 90°), translation precise, retournements
\item \textbf{Manipulation de groupe} : Selection multiple et transformation simultanee
\item \textbf{Annulation/Retablissement} : Historique des operations pour correction d'erreurs
\item \textbf{Alignement assiste} : Outils d'aide a l'alignement manuel
\end{itemize}

\subsubsection{Algorithmes de Suture}

\begin{table}[h]
\centering
\begin{tabular}{|p{4cm}|p{2cm}|p{6cm}|}
\hline
\textbf{Algorithme} & \textbf{Priorite} & \textbf{Cas d'Usage} \\
\hline
Suture automatique SIFT & Critique & Fragments avec texture suffisante et chevauchement \\
\hline
Suture par points & Haute & Controle manuel precis pour cas difficiles \\
\hline
Suture hybride & Moyenne & Combinaison automatique + corrections manuelles \\
\hline
\end{tabular}
\caption{Besoins algorithmiques}
\end{table}

\subsection{Besoins Non-Fonctionnels}

\subsubsection{Performance}

Les exigences de performance ont ete definies en fonction des contraintes operationnelles :

\begin{itemize}
\item \textbf{Temps de reponse} : Interface reactive (< 100ms pour les interactions)
\item \textbf{Throughput} : Traitement d'images jusqu'a 4GB en moins de 10 secondes
\item \textbf{Utilisation memoire} : Optimisation pour fonctionner avec 8GB RAM
\item \textbf{Scalabilite} : Support jusqu'a 50 fragments simultanement
\end{itemize}

\subsubsection{Utilisabilite}

\begin{itemize}
\item \textbf{Courbe d'apprentissage} : Utilisable par un novice en moins de 30 minutes
\item \textbf{Interface intuitive} : Metaphores visuelles familieres
\item \textbf{Feedback utilisateur} : Indication claire de l'etat des operations
\item \textbf{Gestion d'erreurs} : Messages d'erreur comprehensibles et solutions proposees
\end{itemize}

\subsubsection{Fiabilite}

\begin{itemize}
\item \textbf{Robustesse} : Gestion gracieuse des erreurs sans perte de donnees
\item \textbf{Precision} : Alignement avec erreur RMS < 2 pixels
\item \textbf{Reproductibilite} : Resultats identiques pour memes parametres
\end{itemize}

\subsubsection{Compatibilite}

\begin{itemize}
\item \textbf{Systemes d'exploitation} : Windows 10/11 (64-bit) prioritaire
\item \textbf{Formats d'images} : TIFF, SVS, PNG, JPEG
\item \textbf{Interoperabilite} : Export compatible avec QuPath, ImageJ
\end{itemize}

\subsection{Contraintes Techniques}

\subsubsection{Contraintes Materielles}
\begin{itemize}
\item Fonctionnement sur stations de travail standard (8-16 GB RAM)
\item Compatibilite avec cartes graphiques integrees
\item Support des ecrans haute resolution
\end{itemize}

\subsubsection{Contraintes Logicielles}
\begin{itemize}
\item Utilisation de bibliotheques open-source pour la perennite
\item Minimisation des dependances externes
\item Architecture modulaire pour maintenance facilitee
\end{itemize}

\subsubsection{Contraintes Organisationnelles}
\begin{itemize}
\item Deploiement sans privileges administrateur
\item Documentation en francais et anglais
\item Formation utilisateur minimale requise
\end{itemize}

\subsection{Specifications Techniques}

\subsubsection{Architecture Logicielle}
\begin{itemize}
\item \textbf{Pattern architectural} : Model-View-Controller (MVC)
\item \textbf{Langage principal} : Python 3.11+ pour la rapidite de developpement
\item \textbf{Interface graphique} : PyQt6 pour les performances natives
\item \textbf{Traitement d'images} : OpenCV, NumPy, scikit-image
\end{itemize}

\subsubsection{Formats de Donnees}
\begin{itemize}
\item \textbf{Images d'entree} : TIFF pyramidal, SVS, PNG, JPEG
\item \textbf{Metadonnees} : JSON pour la serialisation des transformations
\item \textbf{Export} : TIFF pyramidal, PNG haute qualite
\end{itemize}

\subsubsection{Algorithmes Requis}
\begin{itemize}
\item \textbf{Detection de caracteristiques} : SIFT (Scale-Invariant Feature Transform)
\item \textbf{Correspondance} : FLANN matcher avec test de ratio de Lowe
\item \textbf{Optimisation} : L-BFGS-B pour minimisation non-lineaire
\item \textbf{Transformations} : Matrices de transformation 2D avec interpolation
\end{itemize}

\end{document>