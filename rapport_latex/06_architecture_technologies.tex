\documentclass[12pt,a4paper]{article}
\usepackage[utf8]{inputenc}
\usepackage[french]{babel}
\usepackage{geometry}
\geometry{margin=2.5cm}

\begin{document}

\section{Architecture et Technologies}

\subsection{Choix Technologiques}

\subsubsection{Langage principal : Python 3.11}
\begin{itemize}
\item Ecosysteme riche pour l'imagerie scientifique
\item Bibliotheques specialisees disponibles
\item Developpement rapide et maintenable
\end{itemize}

\subsubsection{Interface graphique : PyQt6}
\begin{itemize}
\item Performance native
\item Widgets avances (OpenGL)
\item Themes personnalisables
\item Multi-plateforme
\end{itemize}

\subsubsection{Traitement d'images}
\begin{itemize}
\item \textbf{OpenCV} : Transformations geometriques
\item \textbf{NumPy} : Calculs matriciels optimises
\item \textbf{scikit-image} : Algorithmes d'analyse d'images
\item \textbf{OpenSlide} : Support formats pyramidaux
\end{itemize}

\subsubsection{Optimisation numerique}
\begin{itemize}
\item \textbf{SciPy} : Algorithmes d'optimisation
\item \textbf{SIFT (OpenCV)} : Detection de caracteristiques
\end{itemize}

\subsection{Architecture Logicielle}

L'architecture suit le pattern Model-View-Controller (MVC) :

\subsubsection{Structure des modules}
\begin{itemize}
\item \textbf{src/core/} : Logique metier
\item \textbf{src/ui/} : Interface utilisateur
\item \textbf{src/algorithms/} : Algorithmes
\item \textbf{src/utils/} : Utilitaires
\end{itemize}

\subsubsection{Patterns de Conception}
\begin{itemize}
\item \textbf{Model-View-Controller (MVC)} : Separation des responsabilites
\item \textbf{Observer Pattern} : Signaux PyQt6 pour la communication
\item \textbf{Strategy Pattern} : Algorithmes de suture interchangeables
\end{itemize}

\end{document}