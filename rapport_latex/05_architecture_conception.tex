\documentclass[12pt,a4paper]{article}
\usepackage[utf8]{inputenc}
\usepackage[french]{babel}
\usepackage{geometry}
\geometry{margin=2.5cm}
\usepackage{setspace}
\onehalfspacing

\begin{document}

\section{Architecture et Conception}

\subsection{Choix Technologiques}

\subsubsection{Justification du Langage Python}

Le choix de Python 3.11 comme langage principal s'appuie sur plusieurs facteurs decisifs :

\begin{itemize}
\item \textbf{Ecosysteme scientifique} : Disponibilite de bibliotheques specialisees (NumPy, SciPy, OpenCV, scikit-image)
\item \textbf{Rapidite de developpement} : Syntaxe claire et productive pour le prototypage rapide
\item \textbf{Communaute active} : Support communautaire important et documentation extensive
\item \textbf{Interoperabilite} : Integration facile avec bibliotheques C/C++ pour les performances critiques
\end{itemize}

\subsubsection{Framework d'Interface : PyQt6}

PyQt6 a ete selectionne pour l'interface graphique en raison de :

\begin{itemize}
\item \textbf{Performances natives} : Rendu accelere materiel via OpenGL
\item \textbf{Widgets avances} : Composants sophistiques pour applications professionnelles
\item \textbf{Themes personnalisables} : Adaptation a l'environnement medical (theme sombre)
\item \textbf{Multi-plateforme} : Potentiel d'extension vers macOS/Linux
\item \textbf{Maturite} : Framework stable avec historique de maintenance long terme
\end{itemize}

\subsubsection{Bibliotheques de Traitement d'Images}

La selection des bibliotheques de traitement s'est basee sur les besoins specifiques :

\begin{itemize}
\item \textbf{OpenCV} : Transformations geometriques optimisees et detection SIFT
\item \textbf{NumPy} : Operations matricielles haute performance
\item \textbf{scikit-image} : Algorithmes d'analyse d'images scientifiques
\item \textbf{OpenSlide} : Support natif des formats d'imagerie medicale pyramidaux
\item \textbf{Pillow} : Manipulation d'images standards et conversion de formats
\end{itemize}

\subsection{Architecture Logicielle}

\subsubsection{Pattern Architectural : MVC}

L'application suit le pattern Model-View-Controller pour assurer une separation claire des responsabilites :

\begin{itemize}
\item \textbf{Model} : Classes Fragment, PointManager encapsulant la logique metier
\item \textbf{View} : Widgets PyQt6 pour l'affichage et l'interaction utilisateur
\item \textbf{Controller} : FragmentManager, MainWindow orchestrant les interactions
\end{itemize}

\subsubsection{Structure Modulaire}

L'architecture modulaire facilite la maintenance et les extensions futures :

\begin{verbatim}
src/
├── core/                    # Logique metier
│   ├── fragment.py         # Modele de donnees fragment
│   ├── fragment_manager.py # Gestionnaire de fragments
│   ├── image_loader.py     # Chargement d'images
│   └── point_manager.py    # Gestion des points etiquetes
├── ui/                     # Interface utilisateur
│   ├── canvas_widget.py    # Canevas principal OpenGL
│   ├── control_panel.py    # Panneau de controle
│   ├── fragment_list.py    # Liste des fragments
│   └── theme.py           # Theme visuel
├── algorithms/             # Algorithmes specialises
│   └── rigid_stitching.py # Suture rigide
└── utils/                 # Utilitaires
    ├── export_manager.py  # Export d'images
    └── pyramidal_exporter.py # Export pyramidal
\end{verbatim}

\subsection{Conception des Composants Principaux}

\subsubsection{Gestionnaire de Fragments (FragmentManager)}

Le FragmentManager constitue le coeur de l'application, gerant :

\begin{itemize}
\item \textbf{Cycle de vie des fragments} : Creation, modification, suppression
\item \textbf{Transformations} : Application et composition des transformations geometriques
\item \textbf{Selection} : Gestion des selections simples et multiples
\item \textbf{Persistance} : Serialisation/deserialisation des etats
\end{itemize}

\subsubsection{Canevas de Rendu (CanvasWidget)}

Le CanvasWidget optimise l'affichage haute performance :

\begin{itemize}
\item \textbf{Rendu OpenGL} : Acceleration materielle pour images volumineuses
\item \textbf{Niveaux de detail (LOD)} : Adaptation automatique selon le zoom
\item \textbf{Culling spatial} : Rendu uniquement des elements visibles
\item \textbf{Cache intelligent} : Mise en cache des transformations couteuses
\end{itemize}

\subsubsection{Chargeur d'Images (ImageLoader)}

L'ImageLoader gere la complexite des formats medicaux :

\begin{itemize}
\item \textbf{Detection automatique} : Identification du format et des capacites
\item \textbf{Chargement progressif} : Support des images pyramidales multi-resolution
\item \textbf{Gestion memoire} : Optimisation pour images volumineuses
\item \textbf{Conversion formats} : Normalisation vers format interne RGBA
\end{itemize}

\subsection{Patterns de Conception Utilises}

\subsubsection{Observer Pattern}

Le systeme de signaux PyQt6 implemente le pattern Observer :

\begin{itemize}
\item \textbf{Decoupage faible} : Composants independants communiquant par signaux
\item \textbf{Reactivite} : Mise a jour automatique de l'interface
\item \textbf{Extensibilite} : Ajout facile de nouveaux observateurs
\end{itemize}

\subsubsection{Strategy Pattern}

Les algorithmes de suture utilisent le pattern Strategy :

\begin{itemize}
\item \textbf{Interchangeabilite} : Selection dynamique d'algorithmes
\item \textbf{Extensibilite} : Ajout facile de nouveaux algorithmes
\item \textbf{Testabilite} : Test independant de chaque strategie
\end{itemize}

\subsubsection{Factory Pattern}

La creation d'exporteurs suit le pattern Factory :

\begin{itemize}
\item \textbf{Abstraction} : Interface commune pour tous les exporteurs
\item \textbf{Flexibilite} : Ajout de nouveaux formats d'export
\item \textbf{Configuration} : Parametrisation selon le format cible
\end{itemize}

\subsection{Gestion des Performances}

\subsubsection{Optimisations Memoire}

\begin{itemize}
\item \textbf{Chargement paresseux} : Images chargees a la demande
\item \textbf{Cache LRU} : Liberation automatique des donnees anciennes
\item \textbf{Compression temporaire} : Reduction de l'empreinte memoire
\item \textbf{Garbage collection} : Gestion explicite de la memoire Python
\end{itemize}

\subsubsection{Optimisations Calcul}

\begin{itemize}
\item \textbf{Vectorisation NumPy} : Operations matricielles optimisees
\item \textbf{Cache de transformations} : Eviter les recalculs identiques
\item \textbf{Parallelisation} : Utilisation des coeurs multiples pour certaines operations
\item \textbf{Algorithmes adaptatifs} : Precision ajustee selon le contexte
\end{itemize}

\subsubsection{Optimisations Interface}

\begin{itemize}
\item \textbf{Double buffering} : Elimination du scintillement
\item \textbf{Rendu differentiel} : Mise a jour des zones modifiees uniquement
\item \textbf{Limitation FPS} : Controle de la frequence de rafraichissement
\item \textbf{Interaction asynchrone} : Interface reactive pendant les calculs
\end{itemize}

\subsection{Securite et Robustesse}

\subsubsection{Gestion d'Erreurs}

\begin{itemize}
\item \textbf{Exceptions typees} : Categorisation des erreurs pour traitement approprie
\item \textbf{Recuperation gracieuse} : Continuation du fonctionnement apres erreurs non-critiques
\item \textbf{Logging detaille} : Tracabilite pour le debogage
\item \textbf{Validation d'entree} : Verification systematique des donnees utilisateur
\end{itemize}

\subsubsection{Integrite des Donnees}

\begin{itemize}
\item \textbf{Sauvegarde automatique} : Prevention de la perte de travail
\item \textbf{Verification coherence} : Controles d'integrite des transformations
\item \textbf{Historique operations} : Tracabilite complete des modifications
\item \textbf{Export securise} : Verification de l'integrite des fichiers exportes
\end{itemize}

\end{document}