\documentclass[12pt,a4paper]{article}
\usepackage[utf8]{inputenc}
\usepackage[french]{babel}
\usepackage{geometry}
\geometry{margin=2.5cm}

\begin{document}

\section{Conclusion}

\subsection{Bilan du Projet}

Ce stage a permis de developper avec succes un outil professionnel de rearrangement et de suture rigide de fragments tissulaires. L'application finale repond aux objectifs fixes et depasse les attentes initiales en termes de fonctionnalites et de qualite.

\subsubsection{Realisations principales}
\begin{itemize}
\item Application desktop complete et fonctionnelle
\item Algorithmes de suture automatique performants
\item Interface utilisateur moderne et intuitive
\item Distribution professionnelle prete pour deploiement
\end{itemize}

\subsection{Impact et Valeur Ajoutee}

\subsubsection{Pour les utilisateurs}
\begin{itemize}
\item Gain de temps significatif (facteur 10x)
\item Precision d'alignement amelioree
\item Workflow standardise et reproductible
\end{itemize}

\subsubsection{Pour l'organisation}
\begin{itemize}
\item Outil differenciant sur le marche
\item Base technologique pour futurs developpements
\item Expertise acquise en imagerie medicale
\end{itemize}

\subsection{Apprentissages Personnels}

Ce stage a ete une experience formatrice exceptionnelle, combinant defis techniques complexes et application pratique dans un domaine scientifique specialise. La realisation d'un projet complet, de la conception a la distribution, a permis d'acquerir une vision globale du developpement logiciel professionnel.

\subsubsection{Points forts developpes}
\begin{itemize}
\item Autonomie dans la resolution de problemes complexes
\item Capacite d'adaptation aux technologies nouvelles
\item Communication technique avec des experts metier
\end{itemize}

\subsubsection{Perspectives de carriere}
Cette experience confirme mon interet pour le developpement d'applications scientifiques et ouvre des perspectives dans les domaines de l'imagerie medicale, de la vision par ordinateur et des outils de recherche.

\subsection{Remerciements}

Je tiens a remercier l'equipe du Scientific Imaging Lab pour leur accueil et leur encadrement, ainsi que tous les utilisateurs qui ont contribue aux tests et a l'amelioration de l'application.

\end{document}