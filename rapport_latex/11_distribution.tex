\documentclass[12pt,a4paper]{article}
\usepackage[utf8]{inputenc}
\usepackage[french]{babel}
\usepackage{geometry}
\geometry{margin=2.5cm}

\begin{document}

\section{Distribution et Deploiement}

\subsection{Packaging avec PyInstaller}

\subsubsection{Defis techniques}
\begin{itemize}
\item Inclusion des DLLs OpenSlide
\item Gestion des dependances conda
\item Optimisation de la taille
\end{itemize}

\subsubsection{Solution implementee}
La solution a consiste a :
\begin{itemize}
\item Creer des hooks PyInstaller personnalises
\item Collecter automatiquement les DLLs necessaires
\item Inclure le package openslide-bin complet
\item Optimiser la taille finale
\end{itemize}

\subsection{Installateur Windows (Inno Setup)}

\subsubsection{Fonctionnalites}
\begin{itemize}
\item Installation sans privileges administrateur
\item Raccourcis bureau et menu demarrer
\item Desinstallation propre
\item Documentation integree
\end{itemize}

\subsubsection{Taille finale}
L'installateur final fait environ 300 MB, incluant toutes les dependances.

\subsection{Documentation Utilisateur}

\subsubsection{Livrables}
\begin{itemize}
\item Guide utilisateur complet (50 pages)
\item Tutoriels video
\item FAQ et depannage
\item Specifications techniques
\end{itemize}

\subsubsection{Formats de distribution}
\begin{itemize}
\item Documentation PDF integree
\item Fichiers texte pour reference rapide
\item Guide de demarrage rapide
\end{itemize}

\subsection{Tests de Distribution}

L'installateur a ete teste sur :
\begin{itemize}
\item Windows 10 (versions 1909, 2004, 21H2)
\item Windows 11 (versions 21H2, 22H2)
\item Machines sans Python/conda installes
\item Differentes configurations materielles
\end{itemize}

\end{document}