\documentclass[12pt,a4paper]{article}
\usepackage[utf8]{inputenc}
\usepackage[french]{babel}
\usepackage{geometry}
\geometry{margin=2.5cm}

\begin{document}

\section{Objectifs du Stage}

\subsection{Objectif Principal}

Developper une application desktop complete permettant la manipulation, l'arrangement et la suture automatique de fragments d'images tissulaires avec une interface utilisateur professionnelle.

\subsection{Objectifs Specifiques}

\subsubsection{Objectifs Techniques}
\begin{itemize}
\item Implementation d'algorithmes de suture rigide
\item Support des formats pyramidaux (TIFF, SVS)
\item Interface graphique haute performance
\item Systeme d'export multi-format
\end{itemize}

\subsubsection{Objectifs Fonctionnels}
\begin{itemize}
\item Manipulation intuitive des fragments (rotation, translation, retournement)
\item Selection et manipulation de groupes
\item Points etiquetes pour alignement precis
\item Previsualisation en temps reel
\end{itemize}

\subsubsection{Objectifs Qualite}
\begin{itemize}
\item Code maintenable et documente
\item Tests et validation
\item Distribution professionnelle (installateur Windows)
\end{itemize}

\subsection{Criteres de Reussite}

Les criteres de reussite definis en debut de stage etaient :
\begin{itemize}
\item Application fonctionnelle avec interface moderne
\item Support des images de grande taille (> 1 GB)
\item Precision d'alignement < 2 pixels RMS
\item Temps de traitement acceptable (< 30 secondes pour 10 fragments)
\item Taux de reussite de suture automatique > 80\%
\end{itemize}

\end{document}