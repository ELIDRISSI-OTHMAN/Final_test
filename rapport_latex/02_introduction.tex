\documentclass[12pt,a4paper]{article}
\usepackage[utf8]{inputenc}
\usepackage[french]{babel}
\usepackage{geometry}
\geometry{margin=2.5cm}

\begin{document}

\section{Introduction}

Ce rapport presente le travail realise durant mon stage au sein du Scientific Imaging Lab, portant sur le developpement d'un outil professionnel de rearrangement et de suture rigide de fragments tissulaires. Cette application desktop, developpee en Python avec PyQt6, repond aux besoins specifiques des laboratoires d'imagerie medicale pour la reconstruction d'images histologiques fragmentees.

Le projet s'inscrit dans le contexte de l'imagerie medicale numerique, ou la manipulation et l'assemblage precis de fragments tissulaires constituent un defi technique majeur pour les chercheurs et praticiens.

\subsection{Contexte du stage}

L'imagerie histologique moderne genere des images de tres haute resolution (gigapixels) de tissus biologiques. Ces images sont souvent fragmentees lors de l'acquisition ou du traitement, necessitant une reconstruction precise pour l'analyse scientifique.

\subsection{Objectifs du projet}

L'objectif principal etait de developper une application desktop complete permettant la manipulation, l'arrangement et la suture automatique de fragments d'images tissulaires avec une interface utilisateur professionnelle.

Les objectifs specifiques incluaient :
\begin{itemize}
\item Implementation d'algorithmes de suture rigide
\item Support des formats pyramidaux (TIFF, SVS)
\item Interface graphique haute performance
\item Systeme d'export multi-format
\end{itemize}

\end{document}