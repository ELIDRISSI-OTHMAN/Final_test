\documentclass[12pt,a4paper]{article}
\usepackage[utf8]{inputenc}
\usepackage[french]{babel}
\usepackage{geometry}
\geometry{margin=2.5cm}
\usepackage{array}

\begin{document}

\section{Analyse des Besoins}

\subsection{Besoins Fonctionnels}

\begin{table}[h]
\centering
\begin{tabular}{|p{4cm}|p{2cm}|p{6cm}|}
\hline
\textbf{Fonctionnalite} & \textbf{Priorite} & \textbf{Description} \\
\hline
Chargement d'images & Critique & Support TIFF pyramidal, SVS, PNG, JPEG \\
\hline
Manipulation fragments & Critique & Translation, rotation, retournement \\
\hline
Suture automatique & Haute & Algorithmes SIFT + optimisation \\
\hline
Points etiquetes & Haute & Alignement manuel precis \\
\hline
Export multi-format & Haute & PNG, TIFF pyramidal \\
\hline
Selection de groupe & Moyenne & Manipulation simultanee \\
\hline
Interface moderne & Moyenne & Theme sombre, ergonomie \\
\hline
\end{tabular}
\caption{Besoins fonctionnels identifies}
\end{table}

\subsection{Besoins Non-Fonctionnels}

\subsubsection{Performance}
\begin{itemize}
\item Chargement d'images > 1 GB
\item Manipulation fluide (> 30 FPS)
\item Memoire optimisee (< 8 GB RAM)
\end{itemize}

\subsubsection{Utilisabilite}
\begin{itemize}
\item Interface intuitive (< 30 min apprentissage)
\item Raccourcis clavier
\item Feedback visuel immediat
\end{itemize}

\subsubsection{Compatibilite}
\begin{itemize}
\item Windows 10/11 (64-bit)
\item Formats d'imagerie standards
\item Distribution sans dependances
\end{itemize}

\subsection{Contraintes Techniques}

\begin{itemize}
\item Utilisation de Python pour la rapidite de developpement
\item Interface native avec PyQt6 pour les performances
\item Support obligatoire des formats pyramidaux
\item Memoire limitee pour les images volumineuses
\end{itemize}

\end{document}