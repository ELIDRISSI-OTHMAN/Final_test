\documentclass[12pt,a4paper]{article}
\usepackage[utf8]{inputenc}
\usepackage[french]{babel}
\usepackage{geometry}
\geometry{margin=2.5cm}

\begin{document}

\section{Developpement de l'Application}

\subsection{Methodologie de Developpement}

\subsubsection{Approche iterative}
Le developpement s'est deroule en 5 phases :
\begin{enumerate}
\item \textbf{Phase 1} : Prototype fonctionnel (chargement + affichage)
\item \textbf{Phase 2} : Manipulation de base (translation, rotation)
\item \textbf{Phase 3} : Algorithmes de suture
\item \textbf{Phase 4} : Interface avancee et export
\item \textbf{Phase 5} : Tests et distribution
\end{enumerate}

\subsubsection{Outils de developpement}
\begin{itemize}
\item \textbf{IDE} : Visual Studio Code avec extensions Python
\item \textbf{Versioning} : Git avec commits atomiques
\item \textbf{Debug} : PyQt6 Developer Tools
\item \textbf{Profiling} : cProfile pour l'optimisation
\end{itemize}

\subsection{Structure des Donnees}

\subsubsection{Classe Fragment}
La classe Fragment encapsule toutes les informations d'un fragment :
\begin{itemize}
\item Donnees d'image (numpy array)
\item Position et transformations
\item Proprietes d'affichage
\item Cache des images transformees
\end{itemize}

\subsubsection{Gestion des Transformations}
\begin{itemize}
\item Cache des images transformees pour les performances
\item Invalidation intelligente lors des modifications
\item Support des rotations arbitraires (pas seulement 90 degres)
\end{itemize}

\subsection{Gestion de la Memoire}

\subsubsection{Optimisations implementees}
\begin{itemize}
\item Chargement paresseux des niveaux pyramidaux
\item Cache LRU pour les images transformees
\item Rendu par niveaux de detail (LOD)
\item Liberation automatique de la memoire
\end{itemize}

\end{document}