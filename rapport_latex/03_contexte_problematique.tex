\documentclass[12pt,a4paper]{article}
\usepackage[utf8]{inputenc}
\usepackage[french]{babel}
\usepackage{geometry}
\geometry{margin=2.5cm}

\begin{document}

\section{Contexte et Problematique}

\subsection{Contexte Scientifique}

L'imagerie histologique moderne genere des images de tres haute resolution (gigapixels) de tissus biologiques. Ces images sont souvent fragmentees lors de l'acquisition ou du traitement, necessitant une reconstruction precise pour l'analyse scientifique.

\subsection{Problematiques Identifiees}

\subsubsection{Problemes techniques}
\begin{itemize}
\item Manipulation manuelle fastidieuse de fragments d'images
\item Absence d'outils specialises pour la suture rigide
\item Formats d'images pyramidaux complexes (TIFF, SVS)
\item Necessite de preserver la precision spatiale
\end{itemize}

\subsubsection{Besoins utilisateurs}
\begin{itemize}
\item Interface intuitive pour les non-informaticiens
\item Support des formats d'imagerie medicale standards
\item Algorithmes de suture automatique et semi-automatique
\item Export vers des formats compatibles avec les outils d'analyse
\end{itemize}

\subsection{Etat de l'Art}

L'analyse des solutions existantes a revele :

\begin{itemize}
\item \textbf{ImageJ/FIJI} : Fonctionnalites limitees pour la suture
\item \textbf{QuPath} : Oriente analyse, pas reconstruction
\item \textbf{Solutions commerciales} : Couteuses et peu flexibles
\item \textbf{Outils academiques} : Souvent incomplets ou obsoletes
\end{itemize}

Cette analyse a confirme le besoin d'un outil specialise combinant facilite d'utilisation et algorithmes avances.

\end{document}